\documentclass[a4paper,10pt]{article} 

% permet de définir des commentaires
\usepackage[utf8]{inputenc}  
\usepackage{minted}
\usepackage{dirtree}
\usepackage[T1]{fontenc} 
\usepackage[latin1]{inputenc} 
\usepackage[french]{babel} 
\usepackage{a4wide} 
\usepackage[dvips]{graphicx}
\pagestyle{empty}
\setlength{\parindent}{0cm}

\begin{document}

\begin{center}
{\Large\bf
\underline{Interprétation et Compilation}}~~~~\\~\\
Compilateur C vers MIPS en Racket\\~\\
Levacher Jimmy\\~\\
31 décembre 2020
\end{center}~

\section{Analyse lexicale}
L’analyse lexicale est la première étape de la lecture d’un code source.\\
Son rôle, est de découper cette chaîne unités lexicales afin de vérifier chaque "mot" du langage.\\\\
Ces mots sont déclarés dans le fichier {\tt lexer.rkt}.\\
Nous retrouvons les mots clefs ({\tt if, else, while, ...}), les ponctuations ({\tt (), \{\}, ;, ...}), les contantes ({\tt Num, Str, Bool, Type}), les opérateurs ({\tt +, -, *, ...}), les identifiants ({\tt nom de variable, function, ...}) et les commentaires ({\tt //, /**/}).\\\\


\begin{figure}[h]
\begin{center}
    \includegraphics[scale=0.8]{images2/1.png}
\end{center}
\caption{\label{fig:1}{\tt Tokens}}
\end{figure}~\\


\begin{minted}{JavaScript}
//JL
\end{minted}
\pagestyle{plain}

\begin{newpage}
\section{Analyse syntaxique}
L’analyse syntaxique est l’étape qui suit l’analyse lexicale et qui précède l’analyse sémantique.\\
Le rôle de cette analyse est de comprendre et vérifier la structure du code source. Cette structure est définie par une grammaire non-contextuelle.
\\
Pour cela, l'outil {\tt parser} en racket sera utilisé dans le fichier {\tt parser.rkt}\\
Ce dernier permet avec les tokens définis dans le lexer, de créer la grammaire et la structure du code.\\

\begin{figure}[h]
\begin{center}
    \includegraphics[scale=0.8]{images2/2.png}
\end{center}
\caption{\label{fig:1}{\tt Tokens}}
\end{figure}~\\

Chaque ligne de notre code sera "comparée" à notre grammaire.\\
Par exemple

\begin{minted}{C}
int a = 10;
\end{minted}~\\
D'après la grammaire, un programme peut-être une {\tt instr}, une {\tt instr} ça peut-être un {\tt type Lident Lassign sexpr Lsemicol}. Et dans notre exemple {\tt int} est un {\tt type}, {\tt a} est un {\tt Lident}, {\tt "="} correspond au token {\tt Lassign}, {\tt 10} est une simple expression {\tt sexpr} et {\tt ";"} correspond bien à notre token {\tt Lsemicol}.

\pagestyle{plain}
\end{newpage}

\begin{newpage}
\section{Analyse sémantique}
Le rôle de l'analyse sémantique, est de comprendre et vérifier le sens du code source, qui est défini par la sémantique du langage.\\
Le but de l’analyse sémantique est de produire un arbre de syntaxe abstraite (AST) sémantiquement valide.\\Cela signifie qu’on doit pouvoir interpréter ou compiler le code représenté par cet AST
sans se soucier de ce qui a déjà été vérifié.\\
Pour cela, il faut d'abord définir une table de hachage pour vérifier les fonctions (add, sub, div, ...) dans le fichier {\tt baselib.rkt}.\\Par exemple, {\tt add} prend une liste de {\tt num} et retourne un {\tt num} qui est le résultat de l'addition de la liste.\\

\begin{figure}[h]
\begin{center}
    \includegraphics[scale=0.8]{images2/3.png}
\end{center}
\caption{\label{fig:1}{\tt Baselib}}
\end{figure}~\\

Une fois fait, le fichier {\tt analyzer.rkt} se chargera d'analyser les expressions et les instructions.

\begin{figure}[h]
\begin{center}
    \includegraphics[scale=0.8]{images2/4.png}
\end{center}
\caption{\label{fig:1}{\tt Baselib}}
\end{figure}~\\

{\tt analyze-instr} va vérifier si la structure de l'instruction passée en argument {\tt match} avec une structure syntaxique.

\pagestyle{plain}
\end{newpage}

\begin{newpage}
\section{Compilation}
Une fois l'analyse sémantique passée, il faut donc compiler le code à l'aide du fichier {\tt compiler.rkt} \\
Le but de la compilation est de générer un fichier avec des insctructions MIPS en fonction de notre AST.\\
Les instructions MIPS sont écrites dans le fichier {\tt mips.rkt}.

\begin{figure}[h]
\begin{center}
    \includegraphics[scale=0.8]{images2/5.png}
\end{center}
\caption{\label{fig:1}{\tt Mips}}
\end{figure}~\\

C'est la fonction {\tt print-instr} qui se chargera d'écrire les instructions dans notre fichier compilé en fonction du hachage de notre {\tt baselib}.\\
C'est cette dernière qui se charge d'appeler la fonction print-instr en fonction de l'instruction "matcher".

\begin{figure}[h]
\begin{center}
    \includegraphics[scale=0.8]{images2/6.png}
\end{center}
\caption{\label{fig:1}{\tt Bselib}}
\end{figure}~\\

Par exemple, pour {\tt add}, les instructions {\tt Lw} et {\tt Add} sont "matchées" dans la fonction print-instr.

\pagestyle{plain}
\end{newpage}


\bibliographystyle{alpha}
\bibliography{locale}
\end{document}
